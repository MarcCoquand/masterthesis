\chapter{Results}\label{results}

The source code for the programs can be found in Appendix~\ref{reasonmlrest}
for the ReasonML implementation and in Appendix~\ref{nodejsrest}
implementation.

\section{Evaluating adherence to SOLID}

We can through an expert analysis analyze the adherence to SOLID guidelines in
Section~\ref{evaluatingmaintainability} of the solution in ReasonML. It is the
resulting code of using the library that is analyzed and not the library itself
as the goal is to find if Functional programming constructs can be used to
enforce an idiomatic solution.  The solution was written by the author in a
``as naive'' approach as possible.  That means that the author did not take any
design guidelines into consideration but created the software in such a way
that it would compile.

\subsubsection{Single Responsibility Principle}

Recall that the Single Responsibility Principle for functional programming
states that all modules should revolve around one type.  The file BookApi.re
contains one product type Book. The modules Encoders and Decoders both use this
type except for one helper function \texttt{int}.  Afterwards we find that all
the functions in the module make use of the book type with Lastly we find that
the module Endpoint all revolve around the type \texttt{route (a)} and that they
all make use book except the handler \texttt{delete}, which is a helper
function for the function \texttt{router}.

\subsubsection{Open/Closed Principle}

OCP, as defined in Section~\ref{openclosed}, that the data structures should be
open for extensions without modifying the previous code.  The book api
functionality can be extended with new endpoints without modifying any of the
original code. The router is implemented as a list of endpoints, thus if the
user wants to add a new endpoint it can append new endpoints to the list.
However it is not possible to extend existing endpoints without modifying the
code.  For example should the user want to prepend so that each uri starts with
\url{/new} then that is not possible, the existing code has to be modified.

\subsubsection{Liskov Segregation Principle}

For this solution Liskov Segregation Principle is not applicable as no
extensions are done.

\subsubsection{Interface Segregation Principle}

Interface segregation principle states that the cardinality should be as low as
possible.  While it is impossible to force the user to have the lowest
cardinality possible the library encourages usage of the lowest possible
cardinality by feeding the arguments into the handler and stating the return
type. So in the BookApi.re that every $specification$ forces a contract on the
function and states that to work they must take the specified arguments which
it will extract from the request with the parser function. So it means that the
cardinality of the handler must be according to the $specification$.

\subsubsection{Dependency Inversion Principle}

Dependency Inversion Principle is about separating the logic from it's
environment.  Since the $specification$ GADT separates the handler from the
specification, it means that should the developer want to change the handler
they can change the argument at one spot. If the developer should want to
change the REST api library, handlers are separated from the
$specfication$. Thus the developer would not need to change any of the logic of
the handlers. Therefore the code follows the dependency inversion principle.

Also due to it's separation it also means that testing the logic of the api 
is easier. In order to test it you 

\subsection{Imperative solution}

Since the solution was developed in a untyped language with no force of
structure it makes sense that the SOLID principles will not be followed. But
for the sake of the argument below we go through them as well.

\begin{description}
	\item[Single Responsibility Principle] The imperative solution breaks SRP
in all handlers by having functions that both parse the requests and performs
the side effects. Demonstrated in the first handler App.get
	\item[Open/Closed Principle] N/A
	\item[Interface Segregation Principle] N/A
	\item[Dependency Inversion Principle] In the imperative solution, it is 
impossible to test the handler in isolation. All systems need to be emulated
such as database and the router.
	\item[Liskov Segregation Principle] N/A
\end{description}

\section{Interviews}

With the method outlined in the previous chapter, the interview was performed
on 4 different people, which is a bit less than the recommended by Norman group
of five people due to difficulty finding enough users (ADD\_REFERENCE
\url{https://www.nngroup.com/articles/why-you-only-need-to-test-with-5-users/}).
It was performed through the use of Skype, a communication
tool\footnote{Skype's website: \url{https://www.skype.com}}. The four respondents were
graduated students of the Engineering Interaction Technology and Design
programme at Umeå university. The programme is a five year degree that combines
education in software engineering with studies in design. The questions were originally
asked in swedish but translated to English by the author. The answers can be found in 
Table~\ref{interviewonetwo} and Table~\ref{interviewthreefour}.

\begin{table}[]
\begin{tabular}{lll}
\hline
\textbf{}                         & \textbf{Person 1}
& \textbf{Person 2}
\\ \hline
\multicolumn{1}{l|}{\textbf{Q1}}  &
\multicolumn{1}{l|}{\begin{tabular}[c]{@{}l@{}}Implemented API that should\\
follow REST. I assumet it's\\ related to CRUD?\end{tabular}}
& \multicolumn{1}{l|}{\begin{tabular}[c]{@{}l@{}}A little bit, REST is an API
with\\ endpoints containing method\\ and headers ensuring you\\ get the right
data.\end{tabular}}
\\ \hline
\multicolumn{1}{l|}{\textbf{Q2}}  &
\multicolumn{1}{l|}{\begin{tabular}[c]{@{}l@{}}Little bit, it should be \\
Javascript and Ocaml \\ combined.\end{tabular}}
& \multicolumn{1}{l|}{No experience}
\\ \hline
\multicolumn{1}{l|}{\textbf{Q3}}  &
\multicolumn{1}{l|}{\begin{tabular}[c]{@{}l@{}}I've implemented an API in\\
Express\end{tabular}}
& \multicolumn{1}{l|}{\begin{tabular}[c]{@{}l@{}}I had a course where I used\\
Express three years ago.\end{tabular}}
\\ \hline
\multicolumn{1}{l|}{\textbf{Q4}}  &
\multicolumn{1}{l|}{\begin{tabular}[c]{@{}l@{}}The title is BookApi.re which\\
describes it quite well. It is\\ a Book api that follows\\ RESTful. It also
manages \\ encoding and decoding.\\ It also checks that the \\ requests are
correctly \\ formatted.\end{tabular}} &
\multicolumn{1}{l|}{\begin{tabular}[c]{@{}l@{}}Encoders and Decoders extract\\
data from the json and\\ I understand the handler\\ functions. \\
However the module Endpoint is unclear.\\ Especially $type\ a.\ route(a)$.\end{tabular}}
\\ \hline
\multicolumn{1}{l|}{\textbf{Q5}}  & \multicolumn{1}{l|}{application json}
& \multicolumn{1}{l|}{Maybe string?}
\\ \hline
\multicolumn{1}{l|}{\textbf{Q6}}  & \multicolumn{1}{l|}{$api/books/:int$}
& \multicolumn{1}{l|}{No clue}
\\ \hline
\multicolumn{1}{l|}{\textbf{Q7}}  & \multicolumn{1}{l|}{application/json}
& \multicolumn{1}{l|}{Content type? Json?}
\\ \hline
\multicolumn{1}{l|}{\textbf{Q8}}  & \multicolumn{1}{l|}{See
Appendix~\ref{putperson1}}
& \multicolumn{1}{l|}{See Appendix~\ref{putperson2}}
\\ \hline
\multicolumn{1}{l|}{\textbf{Q9}}  & \multicolumn{1}{l|}{A book api}
& \multicolumn{1}{l|}{A Book REST api}
\\ \hline
\multicolumn{1}{l|}{\textbf{Q10}} & \multicolumn{1}{l|}{text/plain and
application/json}
& \multicolumn{1}{l|}{text/plain and application/json}
\\ \hline
\multicolumn{1}{l|}{\textbf{Q11}} &
\multicolumn{1}{l|}{\begin{tabular}[c]{@{}l@{}}Content type is json and\\ Accept
might be json?\end{tabular}}
& \multicolumn{1}{l|}{Unsure}
\\ \hline
\end{tabular}
\caption{Raw results interview one and two}
\label{interviewonetwo}
\end{table}

\begin{table}[]
\begin{tabular}{lll}
\hline
\textbf{}                         & \textbf{Person 3}
& \textbf{Person 4}
\\ \hline
\multicolumn{1}{l|}{\textbf{Q1}}  &
\multicolumn{1}{l|}{\begin{tabular}[c]{@{}l@{}}I know what it is. It specifies
\\ how to receive and send \\ information to the client.\end{tabular}}
& \multicolumn{1}{l|}{\begin{tabular}[c]{@{}l@{}}It is used for making HTTP \\
requests and setting up a \\ server using simple methods\\ for
changes.\end{tabular}}
\\ \hline
\multicolumn{1}{l|}{\textbf{Q2}}  & \multicolumn{1}{l|}{I have seen it.}
& \multicolumn{1}{l|}{\begin{tabular}[c]{@{}l@{}}I have no experience but
heard\\ about it\end{tabular}}
\\ \hline
\multicolumn{1}{l|}{\textbf{Q3}}  &
\multicolumn{1}{l|}{\begin{tabular}[c]{@{}l@{}}It is my go to library for\\
writing servers.\end{tabular}}
& \multicolumn{1}{l|}{I have worked with it}
\\ \hline
\multicolumn{1}{l|}{\textbf{Q4}}  &
\multicolumn{1}{l|}{\begin{tabular}[c]{@{}l@{}}Code to encode and decode\\ so
that people can not read the\\ content of the books.\\ \\ Not sure what plain
means in\\ content type.\\ Are modules objects?\end{tabular}} &
\multicolumn{1}{l|}{\begin{tabular}[c]{@{}l@{}}First modules define encoding\\
and decoding json data. \\ \\ Afterwards some helper \\ functions. \\ \\ Lastly
there are endpoints with\\ router defined with different\\ paths and query
parameters.\\ \\ It is a REST api for adding, \\ deleting and modifying
books.\end{tabular}} \\ \hline
\multicolumn{1}{l|}{\textbf{Q5}}  & \multicolumn{1}{l|}{N / A}
& \multicolumn{1}{l|}{\begin{tabular}[c]{@{}l@{}}responds with json and \\
accepts plain.\end{tabular}}
\\ \hline
\multicolumn{1}{l|}{\textbf{Q6}}  & \multicolumn{1}{l|}{$/delete$}
& \multicolumn{1}{l|}{$/api/books$.}
\\ \hline
\multicolumn{1}{l|}{\textbf{Q7}}  & \multicolumn{1}{l|}{application/json}
& \multicolumn{1}{l|}{N/A}
\\ \hline
\multicolumn{1}{l|}{\textbf{Q8}}  & \multicolumn{1}{l|}{See
Appendix~\ref{putperson3}}
& \multicolumn{1}{l|}{See Appendix~\ref{putperson4}}
\\ \hline
\multicolumn{1}{l|}{\textbf{Q9}}  & \multicolumn{1}{l|}{A book api to fetch
books}
& \multicolumn{1}{l|}{A Book REST api}
\\ \hline
\multicolumn{1}{l|}{\textbf{Q10}} & \multicolumn{1}{l|}{string}
& \multicolumn{1}{l|}{\begin{tabular}[c]{@{}l@{}}It can recieve text/plain and
\\ application/json but unsure \\ what it can send\end{tabular}}
\\ \hline
\multicolumn{1}{l|}{\textbf{Q11}} & \multicolumn{1}{l|}{Application/json}
& \multicolumn{1}{l|}{\begin{tabular}[c]{@{}l@{}}Application/json but not
specified\\ what it accepts, assume plain/text.\end{tabular}}
\\ \hline
\end{tabular}
\caption{Raw results interview three and four}
\label{interviewthreefour}
\end{table}

Person 1 and Person 4 understood correctly what the library does. However all
four subjects where slightly confused as to what encoders and decoders were
used for. Not one of the subjects could correctly guess what the accept and
content types in the Javascript solutions. In Q5 for person 3 the question was
omitted as the user could not correctly guess at all what the code was supposed
to do and assumed that it a system for encrypting books.

