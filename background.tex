\chapter{Background}\label{background}

Servers need some protocol to communicate with the clients. One such protocol
is REST. This chapter will introduce REST servers. Afterwards it will
transition into software quality and how software quality is ensured today in
the industry. It will also establish the challenges that arises and pitfalls.

\section{Introduction to REST servers}

Servers are applications that provide functionality for other programs or
devices, called clients. Services are servers that allow sharing data or
resources among clients or to perform a computation.

REST (Representational State Transfer) is a software architecture style that is
used to construct web services. A so called RESTful web service allow requesting
systems to access and manipulate textual representations of web services by
using a set of stateless operations. The architectual constraints of REST are as
follows:

\begin{description}
\item[ Client - Server Architecture ] Separate the concerns between user
interface concerns and data storage concerns.
\item[Statelessness] Each request contains all the information neccessary to
perform a request. State can be handled by cookies on the user side or by using
databases. The server itself contains no state.
\item[Cacheability] As on the World Wide Web, clients and intermediaries can
cache responses. Responses must therefore, implicitly or explicitly, define
themselves as cacheable or not to prevent clients from getting stale or
inappropriate data in response to further requests. Well-managed caching
partially or completely eliminates some client–server interactions, further
improving scalability and performance. 
\item[Layered system] A client can not tell if it is connected to an end server
or some intermeditary server. 
\item[Code on demand] Servers can send functionality of a client via exectuable
code such as javascript. This can be used to send the frontend for example.
\item[Uniform interface] The interface of a RESTful server consists of four
components. The request must specify how it would like the resource to be
represented; that can for example be as JSON, XML or HTTP which are not the
servers internal representation. Servers internal representation is therefore
separated. When the client holds a representation of the resource and metadata
it has enough information to manipulate or delete the resource. Also the REST
server has to, in it's response, specify how the representation for the
resource. This is done using Media type. Some common media types are JSON, HTML
and XML.
\end{description}

A typical HTTP request on a restful server consists of one of the  verbs: GET,
POST, DELETE, PATCH and PUT. They are used as follows:

\begin{description}
\item[GET] Fetches a resource from the server. Does not perform any mutation. 
\item[POST] Update or modify a resource.
\item[PUT] Modify or create a resource.
\item[DELETE] Remove a resource from the server.
\item[PATCH] Changes a resource.
\end{description}

A request will specify a header ``Content-Type'' which contains the media
representation of the request content. For example if the new resource is
represented as Json then content-type will be ``application/json''. It also
specifies a header ``Accept'' which informs which type of representation it
would like to have, for example Html or Json. 

A request will also contain a route for the resource it is requesting. These
requests can also have optional parameters called query parameters. In the
request route:

\begin{lstlisting}
/api/books?author=Mary&published=1995
\end{lstlisting}

the $?$ informs that the request contains query parameters which are optional.
In the example above it specfies that the request wants to access the books
resource with the parameters author as Mary and published as 1995.

When a request has been done the server responds with a status code that
explains the result of the request. The full list of status codes and their
descriptions can be found here:
\url{https://en.wikipedia.org/wiki/List_of_HTTP_status_codes}

\subsection{Implementation concerns for REST apis}

A REST api has to concern themselves with the following:

\begin{itemize}
\item Ensure that the response has the correct status code.
\item Ensure that the correct representation is sent to the client.
\item Parse the route and extract it's parameters. 
\item Parse the query and extract it's parameters.
\item Handle errors if the route or query are badly formatted.
\item Generate the correct response body containing all the resources needed.
\end{itemize}

Every type of error has a specific status code, these need to be set correctly.

\section{Architecture} 

When developing large scale server applications, often the requirements are as
follows:

\begin{itemize}
    \item There is a team of developers
    \item New team members must get productive quickly
    \item The system must be continuously developed and adapt to new
        requirements
    \item The system needs to be continuously tested
    \item System must be able to adapt to new and emerging frameworks
\end{itemize}

Two different approaches to developing these large scale applications are
microservice and monolithic systems. The monolithic system comprises of one big
``top-down'' architecture that dictates what the program should do. This is
simple to develop using some IDE and deploying simply requires deploying some
files to the runtime. 

As the system starts to grow the large monolithic system becomes harder to
understand as the size doubles. As a result, development typically slows down.
Since there are no boundaries, modularity tends to break down and the IDE
becomes slower over time, making it harder to replace parts as needed. Since
redeploying requires the entire application to be replaced and tests becomes
slower; the developer becomes less productive as a result. Since all code is
written in the same environment introducing new technology becomes harder.

In a microservice architecture the program comprises of small entities that each
have their own responsibility. There can be one service for metrics, one that
interacts with the database and one that takes care of frontend. This
decomposition allows the developers to easier understand parts of the system,
scale into autonomous teams, IDE becomes faster since codebases are smaller,
faults become easier to understand as they each break in isolation.  Also
long-term commitment to one stack becomes less and it becomes easier to
introduce a new stack. 

The issue with microservices is that when scaling the complexity becomes harder
to predict. While testing one system in isolation is easier testing the entire
system with all parts together becomes harder. Thus there are different types of
tests that are conducted: unit tests, integration tests and E2E-tests.

\subsection{Unit testing}

Unit testing is a testing method where the individual units of code and
operating procedures are tested to see if they are fit for use. A unit is
informally the smallest testable part of the application. To deal with units
dependence one can use method stubs, mock objects and fakes to test in
isolation. The goal of unit testing is to isolate each part of the programs and
ensure that the individual parts are correct. It also allows for easier
refactoring since it ensures that the individual parts still satisfy their part
of the application.

To create effective unit tests it's important that it's easy to mock examples.
This is usually hindered if the code is dependant on some state since previous
states might affect future states.

\subsection{Integration testing}

Whereas unit testing validates that the individual parts work in isolation;
integration tests make sure that the modules work when combined. The purpose is
to expose faults that occurs when the modules interact with each other.

\subsection{End-2-End Tests}

An End-2-End test (also known as E2E test) is a test that tests an entire
passage through the program, testing multiple components on the way. This
sometimes requires setting up an emulated environment mock environment with fake
variables.

\subsection{Challenges}\label{challenges}

When writing unit tests that depend on some environment, for example fetching a
user from some database, it can be difficult to test without simulating the
environment itself. In such cases one can use dependency injections and mock the
environment with fake data. Dependency injection is a method that substitutes
environment calls and returns data instead. The issue with unit tests is that
even if a feature works well in isolation it does not imply that it will work
well when composed with other functions. It also requires the diligence of the 
developer to enforce that code is written in units and that separation of logic and
environment is done as otherwise E2E-tests and integration-tests need to be used.

The challenge in integration and E2E-tests comes with simulating the entire
environments. Given a server connected to some file storage and a database it
requires setting up a local simulation of that environment to run the tests.
This results in slower execution time for tests and also requires work setting
up the environment. Thus it ends up being costly. Also the bigger the space
that is being tested the less close the test is to actually finding the error,
thus the test ends up finding some error but it can be hard to track it down.

Thus to mitigate these issues the correct architecture needs to be created to
make it easier to test. However if there is nothing forcing the programmer to
develop software in this way it creates the possibility for the programmer to
``cheat'' and create software that is not maintainable. 

\section{SOLID principles}\label{oop}

A poorly written system can lead to rotten design. Martin Robert, a software
engineer, claims that there are four big indicators of rotten design. Rotten
design also leads to problems that were established in
Section~\ref{challenges}, such as making it hard to conduct unit tests. Thus
Martin Robert states that a system should avoid the following.

\begin{description}

\item[ Rigidity ] is the tendency for software to be difficult to
change. This makes it difficult to change non-critical parts of the software and
what can seem like a quick change takes a long time.

\item[ Fragility ] is when the software tends to break when doing
simple changes. It makes the software difficult to maintain, with each fix
introducing new errors.

\item[ Immobility ] is when it is impossible to reuse software from
other projects in the new project. So engineers discover that, even though they
need the same module that was in another project, too much work is required to
decouple and separate the desirable parts.

\item[ Viscosity ] comes in two forms: the viscosity of the environment and the
    viscosity of the design. When making changes to code there are often
        multiple solutions. Some solutions preserve the design of the system and
        some are ``hacks''. The engineer can therefore easily implement an
        unmaintainable solution. The long compile times affect engineers and
        makes them attempt to make changes that do not cause long compile times.
        This leads to viscosity in the environment.

\end{description}

To avoid creating rotten designs, Martin Robert proposes the SOLID guideline.
SOLID mnemonic for five design principles to make software more maintainable,
flexible and understandable. The SOLID guidelines are:

\begin{description}
    \item [Single responsibility principle] Here, responsibility means ``reason
        to change''. Modules and classes should have one reason to change and no
        more.
    \item [Open/Closed principle] States we should write our modules to be
        extended without modification of the original source code.
    \item [Liskov substitution principle] Given a base class and an derived
        class derive, the user of a base class should be able to use the derived
        class and the program should function properly.
    \item [Interface segregation principle] No client should be forced to depend
        on methods it does not use. The general idea is that you want to split
        big interfaces to smaller, specific ones.
    \item [Dependency inversion principle] A strategy to avoid making our source
        code dependent on specific implementations is by using this principle.
        This allows us, if we depend on one third-party module, to swap that
        module for another one should we need to. This can be done by creating
        an abstract interface and then instance that interface with a class that
        calls the third-party operations.~\cite{martinrobert}
\end{description}

Using a SOLID architecture helps making programs that are not as dependent on
the environments which makes them easier to test (swapping the production
environment to a test environment becomes trivial). When investigating the
testability, it is important to look at programs that are written in such a way
that all parts are easy to test. SOLID principles also helps ensuring that
programs are extendable with Interface segregation principle, Open/Closed
principle and Liskov substitution principle. Thus choosing a SOLID architecture
for programs will allow making more testable software. These concepts were
however designed for Object-oriented programming. In Chapter~\ref{method},
these principles will be translated for Functional programming. 

\section{Functional programming for mitigating mistakes}

To mitigate the programmer from making mistakes, some languages feature a type
system. The type system is a compiler check that ensures that the allowed values
are entered. Different strengths exist between various programming languages
with some featuring higher-kinded types (types of types) and other constructs.

It is possible to combine the type system with design patterns to force the
developer to create the right thing. Chapter~\ref{theory} will introduce a REST
framework named Cause, which has been created to force the developer to create
REST compliant servers using Functional programming.

However as Cause makes heavy use of functional programming,  a software
paradigm that is not popular, the hypothesis of this thesis is that readability
might be affected. Thus this thesis aims to investigate the effects of
introducing functional constructions to programmers with little familiarity
with functional programming when it comes to understandability.
Understandability is important to reduce the learning time for programmers and
cut down learning costs. Thus readability of software is an important criteria for 
maintainable software.

Since functional programming is not popular, it seems plausible that
readability would be impacted. When evaluating the readability of code,
companies can use Code reviews. A code review is an activity in which humans
check how well the code can be understood by reading it. Thus similarity it can
be used to evaluate how well users without experience with functional code
understand functional programs. 

\section{Conclusion}

In summary there are four pillars of concern in maintainable software this
thesis aims to address.

\begin{description}
    \item[Testability] Due to rotten design.
    \item[Extendability] Due to rotten design.
    \item[Readability] Multiple factors, this thesis will specifically look at 
		inexperience as a factor of readability.
    \item[Error-proneness] Due to rotten design.
\end{description}

By creating an semi-structured interview, where the programmers is asked open
questions about how the code works it can give insights about the defects of
the software and if there is something fundamental about functional programming
that makes it harder to understand. Adherence to SOLID principles can be used
to avoid rotten design. Thus a good REST library should be understood by
other programmers and follow SOLID principles. Also if the functional software
can enforce that SOLID principles through the type system, it would mean that a
maintainable architecture would be forced by construction.
