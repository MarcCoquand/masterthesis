\title{Comparing Maintainability and Code Quality in Software Paradigms}
\author{
        Marc Coquand\\
        Department of Computer Science\\
        Umeå University\\
}
\date{\today}

\documentclass[12pt]{article}

\begin{document}
\maketitle

\begin{abstract} 

    This study's goal is to compare approaches to functional programs and
    object-oriented programs to find how it affects maintainability and code
    quality.  By looking at 3 cases, we analyze, how does a functional approach
    to software architecture compare to an OOP (Object-oriented programming)
    approach when it comes to maintainability and code quality?

\end{abstract}

\section{Introduction}
This is time for all good men to come to the aid of their party!


\section{Theory}\label{theory}
\subsection{Characteristics of Functional Programming}
Expressions and functions

\subsubsection{Iterator pattern}

\subsection{Object Oriented Programming}\label{oop}
Uses variables, commands and procedures

\subsubsection{SOLID principles}

\section{Methods}\label{methods}

\subsection{Cyclomatic Complexity}

\subsubsection{Cyclomatic Complexity in Functional Programming}

\subsection{Cognitive Dimensions}

\subsection{Case studies}

\subsubsection{Simplified chess game}

Chess is a famous game and assumed that the reader know how it works. Aim
is to implement a simplified variant of it. This is not ordinary chess but a
simplified version:

\begin{itemize} 
    \item Only pawns and horses exist.
    \item You win by removing all the other players pieces.
\end{itemize}

The player should be able to do the following:

\begin{itemize} 
    \item List all available moves for a certain chess piece. 
    \item Move the chess piece to a given space
    \item Switch player after move
    \item Get an overview of the board
    \item Get an error when making invalid moves
\end{itemize}

\subsubsection{to-do List}

A common task in programming is to create some kind of data store with
information. A to-do list is a minimal example of that. It consists of a list of
items that can be used to remember what to do later. The user should be able to:

\begin{itemize}
    \item Create a new item in the to-do list.
    \item Remove an item from the to-do list.
    \item See all items in the to-do list.
    \item Update an item from the to-do list.
\end{itemize}

\subsubsection{Chatbot engine}

Oftentimes when developing applications we have to deal with complex information
input. One of those cases is when we have chat bots. Chat bots are interactive
programs that respond with a text answer to the users input. For this
application we will implement the following:

\begin{itemize}
    \item Interpretor that can handle semi-complex inputs and deal with errors.
    \item Give answers to those inputs in form of text messages.
\end{itemize}    

\section{Results}\label{results}

\section{Conclusions}\label{conclusions}

\section{Limitations}\label{limitations}

\subsection{Improvements to implementation}

\bibliographystyle{abbrv}
\bibliography{thesis}

\end{document}
