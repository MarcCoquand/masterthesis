\section{Functional servers}

\subsection{RESTful servers}


Servers are applications that provide functionality for other programs or
devices, called clients. Services are servers that allow sharing data or
resources among clients or to perform a computation.

REST (Representational State Transfer) is a software architecture style that is
used to construct web services. A so called RESTful web service allow requesting
systems to access and manipulate textual representations of web services by
using a set of stateless operations. The architectual constraints of REST are as
follows:

\begin{description}
\item[ Client - Server Architecture ] Separate the concerns between user
interface concerns and data storage concerns.
\item[Statelessness] Each request contains all the information neccessary to
perform a request. State can be handled by cookies on the user side or by using
databases. The server itself contains no state.
\item[Cacheability] As on the World Wide Web, clients and intermediaries can
cache responses. Responses must therefore, implicitly or explicitly, define
themselves as cacheable or not to prevent clients from getting stale or
inappropriate data in response to further requests. Well-managed caching
partially or completely eliminates some client–server interactions, further
improving scalability and performance. 
\item[Layered system] A client can not tell if it is connected to an end server
or some intermeditary server. 
\item[Code on demand] Servers can send functionality of a client via exectuable
code such as javascript. This can be used to send the frontend for example.
\item[Uniform interface] The interface of a RESTful server consists of four
components. The request must specify how it would like the resource to be
represented; that can for example be as JSON, XML or HTTP which are not the
servers internal representation. Servers internal representation is therefore
separated. When the client holds a representation of the resource and metadata
it has enough information to manipulate or delete the resource. Also the REST
server has to, in it's response, specify how the representation for the
resource. This is done using Media type. Some common media types are JSON, HTML
and XML.
\end{description}

A typical HTTP request on a restful server consists of one of the  verbs: GET,
POST, DELETE, PATCH and PUT. They are used as follows:

\begin{description}
\item[GET] Fetches a resource from the server. Does not perform any mutation. 
\item[POST] Update or modify a resource.
\item[PUT] Modify or create a resource.
\item[DELETE] Remove a resource from the server.
\item[PATCH] Changes a resource.
\end{description}

A request will specify a header ``Content-Type'' which contains the media
representation of the request content. For example if the new resource is
represented as Json then content-type will be ``application/json''. It also
specifies a header ``Accept'' which informs which type of representation it
would like to have, for example Html or Json. 

A request will also contain a route for the resource it is requesting. These
requests can also have optional parameters called query parameters. In the
request route:

\begin{lstlisting}
/api/books?author=Mary&published=1995
\end{lstlisting}

the $?$ informs that the request contains query parameters which are optional.
In the example above it specfies that the request wants to access the books
resource with the parameters author as Mary and published as 1995.


When a request has been done the server responds with a status code that
explains the result of the request. The full list of status codes and their
descriptions can be found here:
\url{https://en.wikipedia.org/wiki/List_of_HTTP_status_codes}

\subsection{Implementation concerns for REST apis}

A REST api has to concern themselves with the following:

\begin{itemize}
\item Ensure that the response has the correct status code.
\item Ensure that the correct representation is sent to the client.
\item Parse the route and extract it's parameters. 
\item Parse the query and extract it's parameters.
\item Handle errors if the route or query are badly formatted.
\item Generate the correct response body containing all the resources needed.
\end{itemize}

Every type of error has a specific status code, these need to be set correctly.

\section{Formal implementation of a server in functional programs}

A server is a function that takes a request of parameter a and transforms it
into a response of parameter a. I.E.  $Server : Request\ a\rightarrow\ Response\
a$. The parameter $a$ is the resource requested by the client.

A $Response$ is a record consisting a status code, a set of headers, a
content type, a function $body: a\rightarrow encoded$, encoding.

\begin{lstlisting}
  type Response a = {
    code: StatusCode,
    headers: Header ,
    contentType: MediaType,
    body: a -> encoded,
    encoding: Encoding.t,
  };
\end{lstlisting}

The body of a response is a function that transforms the resource into it's
requested representation. If the request specifies accept as
\lstinline{application/json} then the body function turns the body into a json
format. 

A $Request$ is 
