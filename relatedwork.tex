\chapter{Related Work}\label{relatedwork}

In the previous chapters we introduced software quality and established why it
is important for software engineers. There exists different metrics for
measuring software quality. One of them is measuring the Lines of Code (LOC) in
the software as a measure of defects.~\cite{defectloc} As this study will
measure different programming languages and software paradigms where syntax is
vastly different, it follows that using LOC to measure if one paradigm is
potentially flawed. Thus another metric to look at is cyclomatic complexity. To
measure software complexity the following metrics were identified:

\begin{description}

    \item [Cyclomatic Complexity] Described further in Chapter~\ref{theory}

    \item [Halsteads metric] A metric that relates to the difficulty of writing
    or understanding code related to operators and operands.~\cite{bergklaas}

    \item [Chidamber and Kemerer Metrics~\cite{chidamber}] A complexity measure
    for Object-oriented programs. Since this study compares functional to
    object-oriented progams this metric is ill suited as it does not allow for
    comparative analysis.

\end{description}

Berg Van Der Klaas explored Halsteads metric and compared OOP and functional
programming. The study notes that the psychological complexity needs to be taken
into account and noted that the use higher order functions may affect results
for functional programs for the Halstead metric. To measure complexity of
software, a program could instead be measured at a lexical level. No
quantitative measure was found for measuring the linguistic structures of the
program and the comprehension. However a qualitative measure called Cognitive
Dimensions was found that could work instead. There has been a thesis done that
looks at Functional programming and OOP when it comes to programming Graphical
User interfaces.~\cite{euguenkiss} Cognitive Dimensions inspects
fourteen different aspects and they are not always applicable for all projects.
The master thesis suggests omitting some which has been taken into account and
is explained further in Chapter~\ref{theory}.
