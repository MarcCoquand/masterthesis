\chapter{Reflections}\label{reflection}

From Chapter~\ref{conclusion} we found that that the functional solutions had
negative impacts in software structural quality. This chapter aims
in Section~\ref{futurework} establish how these results can be improved. It also
aims to conclude this thesis and give suggestions for future research and how it
can be conducted.

\section{Future work}\label{futurework}

SOLID principles places a lot of emphasis on extendability and that modifying
original code is bad practice. It is questionable if these principles are
relevant for strongly typed languages, as the types indicate if there are any
errors in the code, making it easier to refactor.

A lot of the errors in readability might be possible to fix and it might be that
they are not inherent to the language itself. Some were fixed afterwards, the
version of the code in Appendix~\ref{cause} does not feature the problem with
needing to write \texttt{type a. Route(a)}. It would be necessary  retry the
experiment with five more people to find out if the changes made the code more
comprehensible. 

Creating the REST library went through a lot of iterations and a lot of work was
put into making it more readable. GADTs were suprisingly difficult to grok and
the applications were not clear but I hope with this thesis I demonstrated how
they can be useful. I hope using GADTs as a way of constructing inputs to
outputs proves useful for others.

\subsection{Clarifying what is URL and what is not}

Most users were uncertain what the URLs were with only person 1 correctly
assuming that the URI for delete was $/api/books:id$. It might possible to
further clarify what is a URI by wrapping it in a function that takes an
incomplete route, making it clearer for the user what the URL and what is not.
There is also a disconnect since for all other parts of the specification order
does not matter while for the URI it does, this may confuse the user.

\subsection{Extendability for the URIs}

There are some issues still with the server not being as extensible as possible
for it to be OCP compliant. To extend it with the functionality of adding new
details to an existing endpoint would make it more compliant as it would allow
the user to add new functionality to code without recompiling the original
code. 

\subsection{Functional programming for documentation}

When creating software, engineers tend to also document the software for future
use to make it more maintainable. In servers this is usually done manually.
(ADD\_REFERENCE) However if updates are made to the code, the engineer has to
then also manually update the documentation which incurs maintenance costs. It
is plausible that $specfication$ can also be used for documentation. This
should be as simple as creating an evaluator for the $specification$ to
generate the documentation. This would ensure that documentation stays in sync
and minimizes maintenance costs of documentation by automating it.

\subsection{Evaluating effectiveness of SOLID}

The effectiveness of SOLID has not been researched and it would be good to
establish a correlation between software structural quality, defects and SOLID
principles. However due to the expensive nature of software development, this
proves to be difficult as you have to create the same software using different
methods for comparison over multiple years which can lead to biases.

\subsection{Limitations}

There is no way for the REST library to enforce that modification and verbs are
linked. So if a specification specifies that it only works for GET requests,
there is no way to statically enforce that no mutation is done. There might be a
way to statically enforce that handlers with GET requests must use coeffects and
that instead the handler describe what it requires from the
database.(ADD\_REFERENCE \url{http://tomasp.net/coeffects/})


\section{Improvements}

While interviews worked for finding defects in readability, the results could
have been improved by doing the interviews in batches. For those wishing to use
this method, I recommend trying three or four iterations of interviews and
attempt to fix the errors that come up at each set before doing the next set. By
improving the library and server after the interviews, errors that are related
to syntax and semantics which are not inherent to functional programming could
have been eliminated.

It is unclear if SOLID principles are the best principles for maintainable
software in functional programming. For instance it might be worth asking if the
expression problem is truely that big of a problem in practice for software
developers.

I was surprised to find that the readability was impacted negatively and that
despite this most subjects were still capable of extending the code properly
without errors.

\section{Concluding remarks}

I hope this thesis serves to underline the challenges in the software industry
and demonstrate how the software industry can make use of functional programming
to aid creating software that is maintainable in ways that are not possible in
other paradigms. The techniques outlined in this thesis here can be applied to
other protocols in other domains to ensure that certain restraints are held. I
hope also that it demonstrates that will it gives benefits in functional
quality, structural quality might be affected which can potentially lead to
other maintainability problems as the software evolves. My hope is in the future
the software industry aims to write software as a specification and then an
evaluator for that specification rather than first writing a specification on
paper and then implement it as software and manually ensure that the software
follows the specification. 
