\chapter{Conclusion}\label{conclusion}

 The objective of this study is to evaluate how the functional solution impacts
 the software quality. In Chapter~\ref{theory} we showed how functional
 programming can be used to enforce protocols and then using that created a
 library which enables developers to construct correct RESTful apis. Since the
 imperative solution does not enforce that a server is necessarily RESTful, it
 follows that for a user who would typically not create incorrect RESTful apis
 the functional solution can enforce a better software functional quality for
 RESTful apis. However as established in the start of Chapter~\ref{background}
 and~\ref{introduction}, software quality also encompasses the structural
 quality or maintainability. In Section~\ref{backgroundconclusion} we introduced
 the four points of maintainability that needed to be evaluted against the
 library. These were testability, extendability, readability and
 error-proneness. In order to do so we will review the results gathered in
 Chapter~\ref{results} and tie them together with the four points. These four
 points are what determine the software structural quality, which this chapter
 aims to do.

\section{Evaluating the readability}

We find that while everyone understood the Javascript version of the book api,
two (P2 and P3) out of four faced difficulties with the ReasonML version.
Encoders and decoders seemed to have confused P3, as they assumed it related to
cryptography. P2 got confused by the type signature $type\ a.\ route(a)$, which
is necessary for Ocaml to deduce the type signature as it otherwise can not
generalize. It also seems that P4 was incapable of understanding the ReasonML
version and assumed that endpoints were not functions but objects.  3 out of 4
users (P1, P2, P4) were able to extend the code with a new endpoint PUT and P3
was almost able to except that they used the wrong function for handlers.

Further research is needed to find how long it would take for the users to
understand the code. We find that half of the users could understand the new
code base without any form of introduction (P1 and P4). In production it might
be valuable to find a more exact number how long it would take for users to
understand it.  The study indicates that there are costs in readability to the
code for inexperienced users. Arguably these costs seem to be minor and that
after a brief introduction the code would be understood. However more research
is needed to confirm that. It can also be argued that some of those costs can be
mitigated by adding comments and making the changes in Section~\ref{futurework},
but more research is needed to prove this.

\section{Functional programming and SOLID}

In this study we found that functional programming was capable of creating a
library that was able aid in creating Single Responsibility principle, by
encouraging the user to separate the REST specification from how the handler
fetches the data. It also manages to enforce Interface Segregation Principle and
Dependency Inversion Principle. This aids in reducing immobility, fragility and
viscosity. It was inconclusive as to if it enforces Liskov substitution
principle as further work is needed to create examples where this principle is
properly tested. The functional solution also breaks Open/Closed principle for
situations where the user wants to add more details to a specification.

\section{Summary}

Recall that Chapter~\ref{background} introduced the four pillars of concern:
testability, extendability, readability and error-proneness. From these studies
we can conclude that

\begin{description}
    \item[Testability] Software became easier to test compared to an unopiniated
        solution as it managed to invert the control so that unit tests can be
        made for testing specifications which in the imperative solution would
        require an integration test.
	\item[Extendability] No gains were made in extendability when using the 
	functional solution, in fact it has a negative impact due to breaking OCP.
    \item[Error-proneness] The functional solution marginally affects the
        error-proneness as we find that person 3 in Q8 was unable to extend the
        code without making an error.
    \item[Readability] affected negatively since P3 and P2 were unable to
        comprehend the code.
\end{description}

In conclusion, this thesis has demonstrated how functional programming can
enforce good software functional quality but that it seems that in doing so,
software structural quality was negatively impacted in terms of readability and
extendability but positively impacted in testability and error-proneness. 

