\chapter{Introduction: Software paradigms and complexity}\label{introduction}

Different schools of thoughts have different approaches when it comes to
building applications. There is one that is the traditional, object oriented,
procedural way of doing it. Then there is a contender, a functional approach, as
an alternative way to build applications. Functional programming originates from
1936 from Lambda calculus~\cite{Turner} and even though functional programming
is so old, the industry most commonly use Object-oriented, imperative, languages
such as Java. As of today, defects in software are still common place with the
average defect rate being 15- 50 per 10000 lines of code. (ADD\_REFERENCE Steve
McDonnell Code complete) This indicates that the tools used might be inefficient
and improvements can be made. Also with the nature of defects being so common
engineers partly need new tools that decrease defects but also need to ensure
that the for future developers the code is easy to modify so that when defects
show up these can easily be fixed. The software needs to be maintainble to be of
good quality

Software quality can be divided into two different subparts: software functional
quality and software structural quality. Software functional quality reflects
how well our system conforms to given functional requirements or specification
and the degree of which we produce correct software.  To check that the software
is correct, software engineers create tests.~\cite{Pressman:2004:SEP:994110} In
order to create tests, the engineer employs various patterns and tools in the
code to make the code easier to test, such as Object-oriented programming.

Software structural quality refers to how well the software adheres to
non-functional requirements such as robustness and
maintainability.~\cite{Pressman:2004:SEP:994110} Some of the maintainability
aspects, such as readability, is hard to measure quantitatively. By performing
semi-structured interviews, it is possible to investigate how well the code is
understood. 

In this thesis, the focus will be on Functional programming as a potential
solution to increase software quality. Since functional programming is not the
most popular approach to software engineering today, it is worth taking into
consideration how employing functional programming might affect the readability
and maintainability of the software. If no one understands the code, how can
they be expected to maintain the software? The aim is to investigate what makes
software maintainable and can those criterias that make software maintainable be
enforced and with Functional programming. Since software engineering is so
general the focus will be especially on servers. In servers, it is common to use
a protocol called REST to establish communication between servers and clients.
These servers are called RESTful APIs and will be explained further in
Chapter~\ref{background}. As of today, it is up to the developer to manually
ensure that this protocol is followed. Nothing enforces the developer to do it
correctly unless the library implementation enforces it.  Popular frameworks,
such as Express, however do not enforce this protocol.

This thesis will show that it is possible using Functional programming to
implement a server library that forces the user to implement a REST compliant
server which would aid developers in improving the software functional quality.
However functional programming is not commonplace and using such a library might
have a negative impact when it comes to software maintenance and readability,
thus impacting the software structural quality. By establishing the rules in
Object-oriented programming for maintainable software and translating those to
functional programming and by evaluating the readability of code written using
the library, the study aims to find how software structural quality is affected
through the use of functional programming.
