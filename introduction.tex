\chapter{Introduction}\label{introduction}

Different schools of thoughts have different approaches when it comes to
building applications. There is one that is the traditional, object oriented,
procedural way of doing it. Then there is a contender, a functional approach, as
an alternative way to build applications. Functional programming originates from
1936 from Lambda calculus~\cite{Turner} and even though functional programming
is old, the industry most commonly uses Object-oriented, imperative, languages.
~\cite{tiobe2013tiobe} As of today, defects in software are still
commonplace with the average defect rate being 15- 50 per 10000 lines of
code.~\cite{McConnell:2004:CCS:1096143} This indicates that the tools used might
be inefficient and improvements can be made. Also with defects being so common
engineers not only need new tools that decrease defects but also need to ensure
that for future developers, the code is easy to modify so that when defects show
up they can be fixed. The software needs to be maintainable to be of good
quality.

Software quality can be divided into two different subparts: software functional
quality and software structural quality.~\cite{Pressman:2004:SEP:994110}
Software functional quality reflects how well our system conforms to given
functional requirements or specification and the degree of which correct
software is produced.  To check that the software is correct, software engineers
create tests. To create tests, the engineer employs various patterns
and tools in the code to make the code easier to test. These range from
Test-driven development, Object-oriented programming, unit
testing~\cite{beck2003test} to the use of static analysis and logical
proofs. 

Software structural quality refers to how well the software adheres to
non-functional requirements such as robustness and
maintainability.~\cite{Pressman:2004:SEP:994110} Some of the maintainability
aspects, such as readability, are hard to measure quantitatively. By performing
semi-structured interviews, it is possible to investigate how well the code is
understood. 

\section{Objectives}

This thesis aims to investigate how functional programming affects software
quality when compared to imperative programming in server development.  It will
establish what constitutes good functional and structural quality in servers and
then demonstrate how functional programming can be used to construct a library
that forces good software functional quality. 

Since software quality has two aspects, it will investigate afterward the
impacts this functional solution has in software structural quality, which
revolves around maintainability, testability, error-proneness, and readability.
To do so it will construct two identical servers, one written in an imperative
language and one in a functional language.  These will then be compared using
guidelines and interviews to find the impact in structural quality.

In servers, it is common to use a protocol called REST to establish
communication between servers and clients.~\cite{battle2008bridging} These
servers are called RESTful APIs. In popular solutions, such as Express,
developers are not forced to ensure that the server follows REST, which can
potentially lead to errors and maintainability problems. This thesis is outlined
s follows:

\begin{description}
    \item[Chapter~\ref{background}] explains RESTful APIs and establishes the
    challenges as well as current guidelines for ensuring good software
    structural quality in RESTful apis. 
    \item[Chapter~\ref{method}] explains how to quantitatively evaluate the
    software structural quality of REST apis by using interviews and analysing
    how well they follow guidelines. 
    \item[Chapter~\ref{theory}] introduces a library for creating REST servers
    that adhere to the REST protocol by construction, ensuring increased
    software functional quality. The chapter also establishes the guidelines for
    evaluating software structural quality in Functional programming, using
    design patterns from Chapter~\ref{background} as a baseline.   
    \item[Chapter~\ref{results}] explains the results from using the created
    REST library for constructing server and comparing that to another
    imperative solution to compare the differences in software structural
    quality by performing the tests described in Method.
    \item[Chapter~\ref{conclusion}] analyses the impacts of software quality of
    the two solutions and concludes the pros and cons of functional programming
    as a solution for better software quality.
    \item[Chapter~\ref{reflection}] Presents the future work and reflections
    about the thesis.
\end{description}

